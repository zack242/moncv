%-------------------------
% Resume in Latex
% Author : Harshibar (Modified for Zakaria TOZY)
% License : MIT
%------------------------

\documentclass[letterpaper,11pt]{article}

\usepackage{latexsym}
\usepackage[empty]{fullpage}
\usepackage{titlesec}
\usepackage{marvosym}
\usepackage[usenames,dvipsnames]{color}
\usepackage{verbatim}
\usepackage{enumitem}
\usepackage[hidelinks]{hyperref}
\usepackage{fancyhdr}
\usepackage[english]{babel}
\usepackage{tabularx}
\usepackage{fontawesome5}
\usepackage[scale=0.90,lf]{FiraMono}

% light-grey
\definecolor{light-grey}{gray}{0.83}
\definecolor{dark-grey}{gray}{0.3}
\definecolor{text-grey}{gray}{.08}

\DeclareRobustCommand{\ebseries}{\fontseries{eb}\selectfont}
\DeclareTextFontCommand{\texteb}{\ebseries}

% custom underilne
\usepackage{contour}
\usepackage[normalem]{ulem}
\renewcommand{\ULdepth}{1.8pt}
\contourlength{0.8pt}
\newcommand{\myuline}[1]{%
  \uline{\phantom{#1}}%
  \llap{\contour{white}{#1}}%
}

% custom font: helvetica-style
\usepackage{tgheros}
\renewcommand*\familydefault{\sfdefault} 
\usepackage[T1]{fontenc}

\pagestyle{fancy}
\fancyhf{} 
\fancyfoot{}
\renewcommand{\headrulewidth}{0pt}
\renewcommand{\footrulewidth}{0pt}

% Adjust margins
\addtolength{\oddsidemargin}{-0.5in}
\addtolength{\evensidemargin}{0in}
\addtolength{\textwidth}{1in}
\addtolength{\topmargin}{-.5in}
\addtolength{\textheight}{1.0in}

\urlstyle{same}
\raggedbottom
\raggedright
\setlength{\tabcolsep}{0in}

\titleformat{\section}{
    \bfseries \vspace{2pt} \raggedright \large
}{}{0em}{}[\color{light-grey} {\titlerule[2pt]} \vspace{-4pt}]

% Custom commands
\newcommand{\resumeItem}[1]{
  \item\small{
    {#1 \vspace{-1pt}}
  }
}

\newcommand{\resumeSubheading}[4]{
  \vspace{-1pt}\item
    \begin{tabular*}{\textwidth}[t]{l@{\extracolsep{\fill}}r}
      \textbf{#1} & {\color{dark-grey}\small #2}\vspace{1pt}\\
      \textit{#3} & {\color{dark-grey} \small #4}\\
    \end{tabular*}\vspace{-4pt}
}

\newcommand{\resumeProjectHeading}[2]{
    \item
    \begin{tabular*}{\textwidth}{l@{\extracolsep{\fill}}r}
      #1 & {\color{dark-grey} #2}\\
    \end{tabular*}\vspace{-4pt}
}

\newcommand{\resumeSubHeadingListStart}{\begin{itemize}[leftmargin=0in, label={}]}
\newcommand{\resumeSubHeadingListEnd}{\end{itemize}}
\newcommand{\resumeItemListStart}{\begin{itemize}}
\newcommand{\resumeItemListEnd}{\end{itemize}\vspace{0pt}}

\begin{document}

%----------HEADING----------
\begin{center}
    \textbf{\Huge Zakaria TOZY} \\ \vspace{5pt}
    \small \faPhone* \texttt{0617407077} \hspace{1pt} $|$
    \hspace{1pt} \faEnvelope \hspace{2pt} \texttt{zakaria.tozy@icloud.com} \hspace{1pt} $|$ 
    \hspace{1pt} \faLinkedin \hspace{2pt} \texttt{zakaria-tozy} \hspace{1pt} $|$
    \hspace{1pt} \faMapMarker* \hspace{2pt}\texttt{Paris}
    \\ \vspace{-3pt}
\end{center}

% Summary
\vspace{1mm}
{\small \textit{Jeune ingénieur en Data Engineering, avec une double diplomation en Systèmes d'Information et Data Science. Passionné par l'optimisation des données et leur transformation en solutions pratiques.}}

%-----------EXPERIENCE-----------
\section{EXPERIENCE}
  \resumeSubHeadingListStart
    \resumeSubheading
      {AXA Investment Managers}{Août 2023 -- Janvier 2024}
      {Data Engineer / Data Analyst}{Paris}
      \resumeItemListStart
        \resumeItem{Conception et mise en place de \textbf{10+ pipelines ETL} pour l'ingestion de données métiers dans un environnement Cloud}
        \resumeItem{Optimisation des modèles de données, migration vers Python 3.9 et Unity Catalog, gérant \textbf{844 milliards d'euros d'actifs}}
        \resumeItem{Participation aux tests \textbf{Databricks} et création d'un \textbf{PoC} pour l'anonymisation des données clients}
        \resumeItem{Création de tableaux de bord et \textbf{calcul des KPIs} (ME, OV, CIS, FX) pour la vision distribuée/manageriale}
      \resumeItemListEnd

    \resumeSubheading
      {Kalima Blockchain \& IoT}{Avril 2022 -- Août 2022}
      {Software Engineer}{Paris}
      \resumeItemListStart
        \resumeItem{Développement d'un \textbf{explorateur de blockchain} avec Java et LevelDB, traitant \textbf{1000+ transactions/seconde}}
        \resumeItem{Maintenance et ajout de fonctionnalités pour l'outil d'administration blockchain sur \textbf{Node.js}}
        \resumeItem{Conception d'un algorithme d'\textbf{authentification multisignature} pour les DApps}
      \resumeItemListEnd

    \resumeSubheading
      {Le Crédit Lyonnais - LCL}{Janvier 2020 -- Février 2020}
      {IT}{Paris}
      \resumeItemListStart
        \resumeItem{Réalisation d'une étude sur la \textbf{migration d'Office 2010 vers Office 2016}, analyse des fonctionnalités}
        \resumeItem{Exécution de tests sur \textbf{23 000 postes de travail} et développement de procédures d'installation}
      \resumeItemListEnd
  \resumeSubHeadingListEnd

%-----------PROJECTS-----------
\section{PROJECTS}
    \resumeSubHeadingListStart
      \resumeProjectHeading
          {\textbf{Bitcoin Analysis} \textit{(Python, SQL, Airflow, dbt, Snowflake)}} {}
          \resumeItemListStart
            \resumeItem{Développement d'un \textbf{pipeline de données} pour l'analyse des transactions Bitcoin}
            \resumeItem{Optimisation des processus d'analyse de données massives et \textbf{automatisation} du traitement}
          \resumeItemListEnd
          
        \resumeProjectHeading
          {\textbf{Twitter Real-time Analysis} \textit{(Kafka, NLP, KNN)}} {}
          \resumeItemListStart
            \resumeItem{Création d'une solution de \textbf{visualisation et classification en temps réel} des flux Twitter}
            \resumeItem{Application de modèles de \textbf{machine learning} pour la classification des données non structurées}
          \resumeItemListEnd
          
      \resumeProjectHeading
         {\textbf{Energy Disaggregation @Capgemini} \textit{(Numpy, Pandas, Feature Engineering)}} {}
          \resumeItemListStart
            \resumeItem{Extraction de la composante \textbf{thermosensible} de la consommation énergétique en France}
            \resumeItem{Amélioration des \textbf{prévisions énergétiques} avec des techniques avancées de traitement de données}
          \resumeItemListEnd
    \resumeSubHeadingListEnd

%-----------EDUCATION-----------
\section{EDUCATION}
  \resumeSubHeadingListStart
    \resumeSubheading
      {École Centrale d'Électronique (ECE Paris)}{Janvier 2024}
      {Ingénieur en Systèmes d'information et Big data analytics}{Paris}
      \resumeItemListStart
        \resumeItem{\textbf{Top 5\%} de la promotion}
        \resumeItem{Cours : Programmation, Systèmes d'exploitation, Bases de données avancées, DevOps, Sécurité}
      \resumeItemListEnd

    \resumeSubheading
      {Institut Polytechnique de Paris}{Janvier 2024}
      {Master 2 en Data Sciences}{Paris}
      \resumeItemListStart
        \resumeItem{\textbf{Mention Très bien}}
        \resumeItem{Cours : Machine Learning, Big Data Streaming, Cloud Infrastructure, Deep Learning, NLP}
      \resumeItemListEnd
  \resumeSubHeadingListEnd

%-----------SKILLS-----------
\section{SKILLS}
 \begin{itemize}[leftmargin=0in, label={}]
    \small{\item{
     \textbf{Programmation} {: Python, SQL, Java, C++, C\#}\vspace{2pt} \\
     \textbf{Data Engineering} {: PySpark, BigQuery, ETL, Data Pipelines}\vspace{2pt} \\
     \textbf{Cloud \& Databases} {: Databricks, Snowflake, GCP, dbt}\vspace{2pt} \\
     \textbf{Machine Learning} {: Scikit-Learn, Deep Learning, NLP, Computer Vision}\vspace{2pt} \\
     \textbf{DevOps} {: Docker, CI/CD, Git, Scrum methodology}
    }}
 \end{itemize}

% Languages
\vspace{-2mm}
\section{LANGUAGES}
 \begin{itemize}[leftmargin=0in, label={}]
    \small{\item{
     \textbf{Français} {: Langue maternelle}\hspace{12pt}
     \textbf{Anglais} {: Courant (TOEIC : 875)}\hspace{12pt}
     \textbf{Arabe} {: Langue maternelle}
    }}
 \end{itemize}

\end{document} 